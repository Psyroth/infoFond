\documentclass[a4paper,10pt]{article}
\usepackage[francais]{babel}
\usepackage[utf8]{inputenc}
\usepackage[left=2.5cm,top=2cm,right=2.5cm,nohead,nofoot]{geometry}

\linespread{1.1}

\begin{document}

\begin{titlepage}
    \begin{center}
        \textbf{\textsc{UNIVERSIT\'E LIBRE DE BRUXELLES}}\\
        \textbf{\textsc{Faculté des Sciences}}\\
        \textbf{\textsc{Département d'Informatique}}

        \vfill{}\vfill{}

        \begin{center}
            \Huge{Projet Informatique Fondamentale}
        \end{center}
        \Huge{\par}
        \begin{center}
            \large{
                \textsc{Rosette} Arnaud \\
                \textsc{Veranneman} Philippe
            }
        \end{center}
        \Huge{\par}

        \vfill{}\vfill{}

        \vfill{}\vfill{}\enlargethispage{3cm}

        \textbf{Année académique 2013~-~2014}
    \end{center}
\end{titlepage}


\tableofcontents
\pagebreak

\section{Réponses aux parties A}

\subsection{Question 1}

$X_{a,b,c}$ = vrai ssi musicien a joue de l'instrument b dans le groupe c

\subsubsection{Contrainte 1 : Un musicien est dans au moins un groupe (pour un instrument qu'il sait jouer)}


( $instrumentsPlayed(a) \equiv$ la liste des instuments que a sait jouer )


\underline{Pour tout} musicien $a \in M $, \underline{il existe} un groupe $c \in G$ tel que \underline{il existe} un instrument $b \in$ instumentsPlayed(a)  tel que $X_{a,b,c}$


$\bigwedge \limits_{a=0}^{m}$ ( $\bigvee \limits_{c=0}^{k}$ $\bigvee \limits_{b \in instumentsPlayed(a) }$  $X_{a,b,c}$)



\subsubsection{Contrainte 2 : Pas deux fois le même musicien dans un même groupe (au plus un même musicien dans un même groupe)}


\underline{Pour tout} groupe $c \in G$, \underline{pour tout} musicien $a \in M $ tel que \underline{il n'existe pas} i,j $\in I$  tel que $i \neq j$ \underline{et} $X_{a,i,c}$ est vraie
\underline{et} $X_{a,j,c}$ est vraie.


$\bigwedge \limits_{c=0}^{k}$ $\bigwedge \limits_{a=0}^{m}$ $\neg$ [ $\bigvee \limits_{i,j \in (0..n); i\neq j}$ ( $X_{a,i,c}$ $\wedge$ $X_{a,j,c}$ )]


$\equiv$ $\bigwedge \limits_{c=0}^{k}$ $\bigwedge \limits_{a=0}^{m}$  $\bigwedge \limits_{i,j \in (0..n); i\neq j}$ ( $\neg X_{a,i,c}$ $\vee $ $ \neg X_{a,j,c}$ )


Pas besoin car incorporée dans la Contrainte 3.

\subsubsection{Contrainte 3 : Pas deux fois le même musicien dans deux groupes différents et dans un même groupe}

$\delta$ = ensemble des coordonnées [IxG]

 $\bigvee \limits_{(i,j), (i',j') \in \delta ; i\neq i' ou j\neq j'}$ $\bigwedge \limits_{a=0}^{m}$ ( $\neg X_{a,i,j}$ $\vee $ $ \neg X_{a,i',j'}$ )

\subsubsection{Contrainte 4 : Pas 2 musiciens pour le même instrument au sein d'un même groupe}
 
\underline{Pour tout} groupe $c \in G$, \underline{pour tout} instrument $b \in I$ \underline{il n'existe pas} k,l $\in M$  tel que $k \neq l$ \underline{et} $X_{k,b,c}$ est vraie
\underline{et} $X_{l,b,c}$ est vraie. 
 
$\bigwedge \limits_{c=0}^{k}$ $\bigwedge \limits_{b=0}^{n}$ $\neg$ [ $\bigvee \limits_{k,l \in (0..m); k\neq l}$ ( $X_{k,b,c}$ $\wedge$ $X_{l,b,c}$ )]


$\equiv$ $\bigwedge \limits_{c=0}^{k}$ $\bigwedge \limits_{b=0}^{n}$  $\bigwedge \limits_{k,l \in (0..m); k\neq l}$ ( $\neg X_{k,b,c}$ $\vee $ $ \neg X_{l,b,c}$ ) 
 
 
\subsubsection{Contrainte 5 : Un groupe doit être soit complet soit vide} 
 
\underline{Pour tout} groupe $g \in G$, \underline{pour tout} instrument $i \in I$ \underline{il existe} un musicien m tel que $X_{g,m,i}$  \underline{ou} $\neg$ (...)


Nous définissons $actif_g$ : Un groupe est 'actif' lorsqu'il y a au moins un musicien dedans.

$\forall g,m,i$ . $X_{g,m,i} \rightarrow actif_g \equiv \forall g,m,i : \neg X_{g,m,i} \wedge actif_g$

$\bigwedge \limits_{g}$ $\bigwedge \limits_{m}$ $\bigwedge \limits_{i}$ ( $\neg$ $X_{m,i,g}$ $\wedge actif_g $ )
 
$\bigwedge \limits_{g}$ $actif_g \rightarrow (\bigwedge \limits_{i} \bigvee \limits_{m} X_{m,i,g})$  (groupe complet)

$\equiv \bigwedge \limits_{g} ( \neg actif_g \wedge \bigwedge \limits_{i} \bigvee \limits_{m} X_{m,i,g}) $

$ a \vee () $
 
 
\subsection{Question 2}

Les contraintes sont les mêmes que celles de la Question 1 à part qu'un musicien peut maintenant appartenir à plusieurs groupes (selon son 'Max') donc C3 doit être modifiée.
De plus on rajoute une contrainte qui précide qu'il ne peut pas être dans un groupe pour un instrument qu'il ne sait pas jouer.
 
 
\subsubsection{Contrainte 1 : Un musicien est dans au moins un groupe (pour un instrument qu'il sait jouer)}


( $instrumentsPlayed(a) \equiv$ la liste des instuments que a sait jouer )


\underline{Pour tout} musicien $a \in M $, \underline{il existe} un groupe $c \in G$ tel que \underline{il existe} un instrument $b \in$ instumentsPlayed(a)  tel que $X_{a,b,c}$


$\bigwedge \limits_{a=0}^{m}$ ( $\bigvee \limits_{c=0}^{k}$ $\bigvee \limits_{b \in instumentsPlayed(a) }$  $X_{a,b,c}$)



\subsubsection{Contrainte 2 : Pas deux fois le même musicien dans un même groupe (au plus un même musicien dans un même groupe)}


\underline{Pour tout} groupe $c \in G$, \underline{pour tout} musicien $a \in M $ tel que \underline{il n'existe pas} i,j $\in I$  tel que $i \neq j$ \underline{et} $X_{a,i,c}$ est vraie
\underline{et} $X_{a,j,c}$ est vraie.


$\bigwedge \limits_{c=0}^{k}$ $\bigwedge \limits_{a=0}^{m}$ $\neg$ [ $\bigvee \limits_{i,j \in (0..n); i\neq j}$ ( $X_{a,i,c}$ $\wedge$ $X_{a,j,c}$ )]


$\equiv$ $\bigwedge \limits_{c=0}^{k}$ $\bigwedge \limits_{a=0}^{m}$  $\bigwedge \limits_{i,j \in (0..n); i\neq j}$ ( $\neg X_{a,i,c}$ $\vee $ $ \neg X_{a,j,c}$ )


Pas besoin car incorporée dans la Contrainte 3.

\subsubsection{Contrainte 3 : Pas deux fois le même musicien dans deux groupes différents et dans un même groupe}

$\delta$ = ensemble des coordonnées [IxG]

 $\bigvee \limits_{(i,j), (i',j') \in \delta ; i\neq i' ou j\neq j'}$ $\bigwedge \limits_{a=0}^{m}$ ( $\neg X_{a,i,j}$ $\vee $ $ \neg X_{a,i',j'}$ )
!!!!!!!!!! a modifier? ou enlever? !!!!!!!!!!

\subsubsection{Contrainte 4 : Pas 2 musiciens pour le même instrument au sein d'un même groupe}
 
\underline{Pour tout} groupe $c \in G$, \underline{pour tout} instrument $b \in I$ \underline{il n'existe pas} k,l $\in M$  tel que $k \neq l$ \underline{et} $X_{k,b,c}$ est vraie
\underline{et} $X_{l,b,c}$ est vraie. 
 
$\bigwedge \limits_{c=0}^{k}$ $\bigwedge \limits_{b=0}^{n}$ $\neg$ [ $\bigvee \limits_{k,l \in (0..m); k\neq l}$ ( $X_{k,b,c}$ $\wedge$ $X_{l,b,c}$ )]


$\equiv$ $\bigwedge \limits_{c=0}^{k}$ $\bigwedge \limits_{b=0}^{n}$  $\bigwedge \limits_{k,l \in (0..m); k\neq l}$ ( $\neg X_{k,b,c}$ $\vee $ $ \neg X_{l,b,c}$ ) 


\subsubsection{Contrainte 5 : Un groupe doit être soit complet soit vide} 
 
\underline{Pour tout} groupe $g \in G$, \underline{pour tout} instrument $i \in I$ \underline{il existe} un musicien m tel que $X_{g,m,i}$  \underline{ou} $\neg$ (...)


Nous définissons $actif_g$ : Un groupe est 'actif' lorsqu'il y a au moins un musicien dedans.

$\forall g,m,i$ . $X_{g,m,i} \rightarrow actif_g \equiv \forall g,m,i : \neg X_{g,m,i} \wedge actif_g$

$\bigwedge \limits_{g}$ $\bigwedge \limits_{m}$ $\bigwedge \limits_{i}$ ( $\neg$ $X_{m,i,g}$ $\wedge actif_g $ )
 
$\bigwedge \limits_{g}$ $actif_g \rightarrow (\bigwedge \limits_{i} \bigvee \limits_{m} X_{m,i,g})$  (groupe complet)

$\equiv \bigwedge \limits_{g} ( \neg actif_g \wedge \bigwedge \limits_{i} \bigvee \limits_{m} X_{m,i,g}) $

$ a \vee () $


\subsubsection{Contrainte 6 : Un musicien ne peut pas être dans un groupe pour un instrument qu'il ne sait pas jouer}


$\bigwedge \limits_{(i^l, j^l) \in \delta^l}$ $ \neg (\bigvee \limits_{m} \bigwedge \limits_{r=0}^{l} X_{m,i^l,j^l}) $
$\equiv \bigwedge \limits_{(i^l, j^l) \in \delta^l} \bigwedge \limits_{m} \bigvee \limits_{r=0}^{l} \neg X_{m,i^l,j^l}$

\underline{Pour tout} musicien $m \in M$, \underline{pour tout} groupe $g \in G$ \underline{pour tout} instrument $i \notin instrumentsPlayed(m)$

$\bigwedge \limits_{m}$ $\bigwedge \limits_{g}$ $\bigwedge \limits_{i \notin instrumentsPlayed}$  $\neg$ $X_{m,i,g}$
 
\end{document}
