\documentclass[a4paper,10pt]{article}
\usepackage[francais]{babel}
\usepackage[utf8]{inputenc}
\usepackage[left=2.5cm,top=2cm,right=2.5cm,nohead,nofoot]{geometry}
\usepackage{hyperref}

\linespread{1.1}

\begin{document}

\begin{titlepage}
    \begin{center}
        \textbf{\textsc{UNIVERSIT\'E LIBRE DE BRUXELLES}}\\
        \textbf{\textsc{Faculté des Sciences}}\\
        \textbf{\textsc{Département d'Informatique}}

        \vfill{}\vfill{}

        \begin{center}
            \Huge{Projet Informatique Fondamentale}
        \end{center}
        \Huge{\par}
        \begin{center}
            \large{
                \textsc{Rosette} Arnaud \\
                \textsc{Veranneman} Philippe
            }
        \end{center}
        \Huge{\par}

        \vfill{}\vfill{}

        \vfill{}\vfill{}\enlargethispage{3cm}

        \textbf{Année académique 2013~-~2014}
    \end{center}
\end{titlepage}


\tableofcontents
\pagebreak


\section{Question 1}

\subsection{Définition de variables : }

$X_{a,b,c}$ = vrai ssi musicien a joue de l'instrument b dans le groupe c.

$P_{a,c}$ = vrai ssi le musicien a joue dans le groupe c.

$A_c$ = vrai ssi le groupe c est actif (un groupe est actif lorsqu'il y a au moins un musicien dans le groupe)

$instrumentsPlayed(a)$ = l'ensemble des instruments que a sait jouer

\subsection{Nouvelles contraintes : }

\subsubsection{Contrainte : Définition de $P_{a,c}$ par rapport a $X_{a,b,c}$}

\underline{Pour tout} musicien $a \in M $, \underline{pour tout} un groupe $c \in G$ , 

(\underline{il existe} un instrument $b \in$ I  tel que $X_{a,b,c}$)
\underline{équivaut à} $P_{a,c}$

$\bigwedge \limits_{a=1}^{m}  \bigwedge \limits_{c=1}^{k}$ [ ( $\bigvee \limits_{b=1}^{n}  X_{a,b,c}$) $\leftrightarrow P_{a,c}$]

$\equiv  \bigwedge \limits_{a=1}^{m}  \bigwedge \limits_{c=1}^{k}$ [ $ ( (\bigvee \limits_{b=1}^{n}  X_{a,b,c}) \rightarrow P_{a,c} ) \wedge
(P_{a,c} \rightarrow  (\bigvee \limits_{b=1}^{n}  X_{a,b,c}) )$ ]

$\equiv  \bigwedge \limits_{a=1}^{m}  \bigwedge \limits_{c=1}^{k}$ [ $ ( \neg (\bigvee \limits_{b=1}^{n}  X_{a,b,c}) \vee P_{a,c} ) \wedge
 (\neg P_{a,c} \vee  (\bigvee \limits_{b=1}^{n}  X_{a,b,c}) )$ ]
 
$\equiv  \bigwedge \limits_{a=1}^{m}  \bigwedge \limits_{c=1}^{k}$ [ $ ( ( \bigwedge \limits_{b=1}^{n}  \neg X_{a,b,c}) \vee P_{a,c} ) \wedge
 (\neg P_{a,c} \vee  (\bigvee \limits_{b=1}^{n}  X_{a,b,c}) )$ ]
 
$\equiv  \bigwedge \limits_{a=1}^{m}  \bigwedge \limits_{c=1}^{k}$ [ $ \bigwedge \limits_{b=1}^{n}  ( \neg X_{a,b,c} \vee P_{a,c} ) \wedge
 (\neg P_{a,c} \vee  (\bigvee \limits_{b=1}^{n}  X_{a,b,c}) )$ ] 

 
\subsubsection{Contrainte : Un musicien est dans au moins un groupe (pour un instrument qu'il sait jouer)}

\underline{Pour tout} musicien $a \in M $, \underline{il existe} un groupe $c \in G$ , \underline{il existe} un instrument $b \in$ instumentsPlayed(a)  tel que $X_{a,b,c}$


$\bigwedge \limits_{a=1}^{m}$ ( $\bigvee \limits_{c=1}^{k}$ $\bigvee \limits_{b \in instumentsPlayed(a) }$  $X_{a,b,c}$)



\subsubsection{Contrainte : Au plus un même musicien dans un même groupe}


\underline{Pour tout} groupe $c \in G$, \underline{pour tout} musicien $a \in M $, \underline{il n'existe pas} deux instruments différents b,b' $\in I$  tel que $b < b'$ et $X_{a,b,c}$
\underline{et} $X_{a,b',c}$.


$\bigwedge \limits_{c=1}^{k}$ $\bigwedge \limits_{a=1}^{m}$ $\neg$ [ $\bigvee \limits_{b,b' \in (1..n); b < b'}$ ( $X_{a,b,c}$ $\wedge$ $X_{a,b',c}$ )]


$\equiv$ $\bigwedge \limits_{c=1}^{k}$ $\bigwedge \limits_{a=1}^{m}$  $\bigwedge \limits_{b,b' \in (1..n); b < b'}$ ( $\neg X_{a,b,c}$ $\vee $ $ \neg X_{a,b',c}$ )


\subsubsection{Contrainte : Un musicien est dans au plus un groupe}

\underline{Pour tout} musicien $a \in M $, \underline{il n'existe pas} deux groupes différents $c,c' \in G$, tel que $c < c'$, $P_{a,c}$ \underline{et} $P_{a,c'}$.

 $\bigwedge \limits_{a=1}^{m}$ $ \neg$ [ $\bigvee \limits_{c,c' \in (1..k); c < c'}$ ( $P_{a,c}$ $\wedge$ $P_{a,c'}$ )]
 
$\equiv$ $\bigwedge \limits_{a=1}^{m}$ $\bigwedge \limits_{c,c' \in (1..k); c < c'}$  ( $\neg P_{a,c} \vee \neg P_{a,c'} $ )


\subsubsection{Contrainte : Au plus un musicien pour le même instrument au sein d'un même groupe}
 
\underline{Pour tout} groupe $c \in G$, \underline{pour tout} instrument $b \in I$ \underline{il n'existe pas} a,a' $\in M$  tel que $a < a'$ et $X_{a,b,c}$
\underline{et} $X_{a',b,c}$ . 
 
$\bigwedge \limits_{c=1}^{k}$ $\bigwedge \limits_{b=1}^{n}$ $\neg$ [ $\bigvee \limits_{a,a' \in (1..m); a < a'}$ ( $X_{a,b,c}$ $\wedge$ $X_{a',b,c}$ )]


$\equiv$ $\bigwedge \limits_{c=1}^{k}$ $\bigwedge \limits_{b=1}^{n}$  $\bigwedge \limits_{a,a' \in (1..m); a < a'}$ ( $\neg X_{a,b,c}$ $\vee $ $ \neg X_{a',b,c}$ ) 
 
 
\subsubsection{Contrainte : Un groupe doit être soit complet soit vide} 
 
\begin{enumerate}

 \item Si un musicien joue d'un instrument dans un groupe, alors ce groupe est actif.
 
    \underline{Pour tout} groupe $c \in G$, \underline{pour tout} musicien $a \in M$ \underline{pour tout} instrument $b \in I$  \underline{si} $X_{a,b,c}$ \underline{alors} $A_c$.
    
    $ \bigwedge \limits_{c=1}^{k} \bigwedge \limits_{a=1}^{m} \bigwedge \limits_{b=1}^{n}$ ( $ X_{a,b,c} \rightarrow A_c$ )
    
    $\equiv \bigwedge \limits_{c=1}^{k} \bigwedge \limits_{a=1}^{m} \bigwedge \limits_{b=1}^{n}$ ( $ \neg  X_{a,b,c} \vee A_c$ )

 
 
 \item \underline{Si} un groupe est actif, \underline {alors} ce groupe est complet.
 
    \underline{Pour tout} groupe $c \in G$, \underline{si} $A_c$ \underline{alors} \underline{pour tout} instrument $b \in I$ \underline{il existe} un musicien $a \in M$
    tel que $X_{a,b,c}$.
    
    
    $ \bigwedge \limits_{c=1}^{k} [ A_c \rightarrow (  \bigwedge \limits_{b=1}^{n} \bigvee \limits_{a=1}^{m} X_{a,b,c} ) ]$
    
    $\equiv \bigwedge \limits_{c=1}^{k} [ \neg A_c \vee (  \bigwedge \limits_{b=1}^{n} \bigvee \limits_{a=1}^{m} X_{a,b,c} ) ]$
    
    $\equiv \bigwedge \limits_{c=1}^{k} \bigwedge \limits_{b=1}^{n}  [ \neg A_c \vee ( \bigvee \limits_{a=1}^{m} X_{a,b,c} ) ]$
 
\end{enumerate}

 

 
 
\section{Question 2}

Les 5 premières contraintes de la Question 2 sont les mêmes que celles de la Question 1 à part qu'un musicien peut maintenant appartenir à plusieurs groupes (selon son 'Max') donc la contrainte 'Un musicien est dans au plus un groupe'  doit être modifiée en 'Un musicien est dans au plus n groupes'.

De plus on rajoute une contrainte qui précise que un musicien ne peut pas être dans un groupe pour un instrument qu'il ne sait pas jouer.

\subsection{Contraintes reprises de la question précédente : }
 
\subsubsection{Contrainte : Définition de $P_{a,c}$ par rapport a $X_{a,b,c}$}

\underline{Pour tout} musicien $a \in M $, \underline{pour tout} un groupe $c \in G$ , 

(\underline{il existe} un instrument $b \in$ I  tel que $X_{a,b,c}$)
\underline{équivaut à} $P_{a,c}$

$\bigwedge \limits_{a=1}^{m}  \bigwedge \limits_{c=1}^{k}$ [ ( $\bigvee \limits_{b=1}^{n}  X_{a,b,c}$) $\leftrightarrow P_{a,c}$]

$\equiv  \bigwedge \limits_{a=1}^{m}  \bigwedge \limits_{c=1}^{k}$ [ $ ( (\bigvee \limits_{b=1}^{n}  X_{a,b,c}) \rightarrow P_{a,c} ) \wedge
(P_{a,c} \rightarrow  (\bigvee \limits_{b=1}^{n}  X_{a,b,c}) )$ ]

$\equiv  \bigwedge \limits_{a=1}^{m}  \bigwedge \limits_{c=1}^{k}$ [ $ ( \neg (\bigvee \limits_{b=1}^{n}  X_{a,b,c}) \vee P_{a,c} ) \wedge
 (\neg P_{a,c} \vee  (\bigvee \limits_{b=1}^{n}  X_{a,b,c}) )$ ]
 
$\equiv  \bigwedge \limits_{a=1}^{m}  \bigwedge \limits_{c=1}^{k}$ [ $ ( ( \bigwedge \limits_{b=1}^{n}  \neg X_{a,b,c}) \vee P_{a,c} ) \wedge
 (\neg P_{a,c} \vee  (\bigvee \limits_{b=1}^{n}  X_{a,b,c}) )$ ]
 
$\equiv  \bigwedge \limits_{a=1}^{m}  \bigwedge \limits_{c=1}^{k}$ [ $ \bigwedge \limits_{b=1}^{n}  ( \neg X_{a,b,c} \vee P_{a,c} ) \wedge
 (\neg P_{a,c} \vee  (\bigvee \limits_{b=1}^{n}  X_{a,b,c}) )$ ] 

 
\subsubsection{Contrainte : Un musicien est dans au moins un groupe (pour un instrument qu'il sait jouer)}


\underline{Pour tout} musicien $a \in M $, \underline{il existe} un groupe $c \in G$ , \underline{il existe} un instrument $b \in$ instumentsPlayed(a)  tel que $X_{a,b,c}$


$\bigwedge \limits_{a=1}^{m}$ ( $\bigvee \limits_{c=1}^{k}$ $\bigvee \limits_{b \in instumentsPlayed(a) }$  $X_{a,b,c}$)



\subsubsection{Contrainte : Au plus un même musicien dans un même groupe}


\underline{Pour tout} groupe $c \in G$, \underline{pour tout} musicien $a \in M $, \underline{il n'existe pas} deux instruments différents b,b' $\in I$  tel que $b < b'$ et $X_{a,b,c}$
\underline{et} $X_{a,b',c}$.


$\bigwedge \limits_{c=1}^{k}$ $\bigwedge \limits_{a=1}^{m}$ $\neg$ [ $\bigvee \limits_{b,b' \in (1..n); b < b'}$ ( $X_{a,b,c}$ $\wedge$ $X_{a,b',c}$ )]


$\equiv$ $\bigwedge \limits_{c=1}^{k}$ $\bigwedge \limits_{a=1}^{m}$  $\bigwedge \limits_{b,b' \in (1..n); b < b'}$ ( $\neg X_{a,b,c}$ $\vee $ $ \neg X_{a,b',c}$ )


\subsubsection{Contrainte : Au plus un musicien pour le même instrument au sein d'un même groupe}
 
\underline{Pour tout} groupe $c \in G$, \underline{pour tout} instrument $b \in I$ \underline{il n'existe pas} a,a' $\in M$  tel que $a < a'$ et $X_{a,b,c}$
\underline{et} $X_{a',b,c}$ . 
 
$\bigwedge \limits_{c=1}^{k}$ $\bigwedge \limits_{b=1}^{n}$ $\neg$ [ $\bigvee \limits_{a,a' \in (1..m); a < a'}$ ( $X_{a,b,c}$ $\wedge$ $X_{a',b,c}$ )]


$\equiv$ $\bigwedge \limits_{c=1}^{k}$ $\bigwedge \limits_{b=1}^{n}$  $\bigwedge \limits_{a,a' \in (1..m); a < a'}$ ( $\neg X_{a,b,c}$ $\vee $ $ \neg X_{a',b,c}$ ) 
 
 
\subsubsection{Contrainte : Un groupe doit être soit complet soit vide} 
 
\begin{enumerate}

 \item Si un musicien joue d'un instrument dans un groupe, alors ce groupe est actif.
 
    \underline{Pour tout} groupe $c \in G$, \underline{pour tout} musicien $a \in M$ \underline{pour tout} instrument $b \in I$  \underline{si} $X_{a,b,c}$ \underline{alors} $A_c$.
    
    $ \bigwedge \limits_{c=1}^{k} \bigwedge \limits_{a=1}^{m} \bigwedge \limits_{b=1}^{n}$ ( $ X_{a,b,c} \rightarrow A_c$ )
    
    $\equiv \bigwedge \limits_{c=1}^{k} \bigwedge \limits_{a=1}^{m} \bigwedge \limits_{b=1}^{n}$ ( $ \neg  X_{a,b,c} \vee A_c$ )

 
 
 \item \underline{Si} un groupe est actif, \underline {alors} ce groupe est complet.
 
    \underline{Pour tout} groupe $c \in G$, \underline{si} $A_c$ \underline{alors} \underline{pour tout} instrument $b \in I$ \underline{il existe} un musicien $a \in M$
    tel que $X_{a,b,c}$.
    
    
    $ \bigwedge \limits_{c=1}^{k} [ A_c \rightarrow (  \bigwedge \limits_{b=1}^{n} \bigvee \limits_{a=1}^{m} X_{a,b,c} ) ]$
    
    $\equiv \bigwedge \limits_{c=1}^{k} [ \neg A_c \vee (  \bigwedge \limits_{b=1}^{n} \bigvee \limits_{a=1}^{m} X_{a,b,c} ) ]$
    
    $\equiv \bigwedge \limits_{c=1}^{k} \bigwedge \limits_{b=1}^{n}  [ \neg A_c \vee ( \bigvee \limits_{a=1}^{m} X_{a,b,c} ) ]$
 
\end{enumerate}

\subsection{Nouvelles contraintes : }

\subsubsection{Contrainte : Un musicien est dans au plus n groupes ($Max_a$)}

\underline{Pour tout} musicien $a \in M$, \underline{il n'existe pas} des groupes $(c^1,c^2,..c^{n+1}) \in G$ tel que $c^1 < c^2 <...<c^{n+1}$ tel que $P_{a,c^1}$ \underline{et} $P_{a,c^2}$
\underline{et} ... \underline{et} $P_{a,c^{n+1}}$
 
$ \bigwedge \limits_{a=1}^{m} \neg ( \bigvee \limits_{c^1,c^2,..c^{n+1} \in G; c^1 < c^2 <...<c^{n+1}} ( P_{a,c^1} \wedge P_{a,c^2} \wedge ... \wedge P_{a,c^{n+1}} ))$

\subsubsection{Contrainte : Un musicien ne peut pas apparaître dans un groupe pour un instrument qu'il ne sait pas jouer}


\underline{Pour tout} musicien $a \in M $, \underline{pour tout} groupe $c \in G$ , \underline{pour tout} instrument $b \notin $ instumentsPlayed(a)  tel que $\neg X_{a,b,c}$


$\bigwedge \limits_{a=1}^{m}$  $\bigwedge \limits_{c=1}^{k}$ $\bigwedge \limits_{b \notin instumentsPlayed(a) }$  $\neg X_{a,b,c}$


\subsection{ Le nombre de clauses que l'encodage de la contrainte 'au plus k' génère}


Le nombre de clauses que l'encodage va générer se calcule par $k \choose Max_a + 1$ c'est à dire $\frac {k!}{Max_a ! (k - Max_a) !} $. En effet il s'agit du nombre de sous ensembles
de littéraux de taille $Max_a$ dans l'ensemble des littéraux de taille k (car pour chaque musicien il existe $k*C_{a,c}$ littéraux).

Si nous prenons par exemple la valeur $Max_a = 2 $ et $k=3$. Cela veut dire qu'un musicien ne peut apparaître que dans 2 groupes au maximum parmi les 3 groupes disponibles.
Il peut donc se trouver dans : $(g1 \vee g2) \wedge (g1 \vee g3) \wedge (g2 \vee g3).$ 

\section{Question 3}

Les 6 premières contraintes de la Question 3 sont les mêmes que celles de la Question 2. Nous modifions la contrainte 'Un musicien est dans au moins un groupe (pour un instrument qu'il sait jouer)' qui devient 'Un musicien est dans au moins un groupe (pour un instrument qu'il sait jouer) \underline{ou pour chanter}'

De plus, on rajoute les contraintes qui précisent qu'il y a au minimum un chanteur par groupe et qu'il n'y a que les musiciens qui maitrisent le chant qui peuvent chanter.

\subsection{Définition de variables : }

$ C_{a,c} $ = vrai ssi le musicien a chante dans le groupe c.

$M'\subseteq M$ = ensemble des musiciens qui maitrisent le chant.

\subsection{Contraintes reprises de la question précédente : }

\subsubsection{Contrainte : Définition de $P_{a,c}$ par rapport a $X_{a,b,c}$}

\underline{Pour tout} musicien $a \in M $, \underline{pour tout} un groupe $c \in G$ , 

(\underline{il existe} un instrument $b \in$ I  tel que $X_{a,b,c}$)
\underline{équivaut à} $P_{a,c}$

$\bigwedge \limits_{a=1}^{m}  \bigwedge \limits_{c=1}^{k}$ [ ( $\bigvee \limits_{b=1}^{n}  X_{a,b,c}$) $\leftrightarrow P_{a,c}$]

$\equiv  \bigwedge \limits_{a=1}^{m}  \bigwedge \limits_{c=1}^{k}$ [ $ ( (\bigvee \limits_{b=1}^{n}  X_{a,b,c}) \rightarrow P_{a,c} ) \wedge
(P_{a,c} \rightarrow  (\bigvee \limits_{b=1}^{n}  X_{a,b,c}) )$ ]

$\equiv  \bigwedge \limits_{a=1}^{m}  \bigwedge \limits_{c=1}^{k}$ [ $ ( \neg (\bigvee \limits_{b=1}^{n}  X_{a,b,c}) \vee P_{a,c} ) \wedge
 (\neg P_{a,c} \vee  (\bigvee \limits_{b=1}^{n}  X_{a,b,c}) )$ ]
 
$\equiv  \bigwedge \limits_{a=1}^{m}  \bigwedge \limits_{c=1}^{k}$ [ $ ( ( \bigwedge \limits_{b=1}^{n}  \neg X_{a,b,c}) \vee P_{a,c} ) \wedge
 (\neg P_{a,c} \vee  (\bigvee \limits_{b=1}^{n}  X_{a,b,c}) )$ ]
 
$\equiv  \bigwedge \limits_{a=1}^{m}  \bigwedge \limits_{c=1}^{k}$ [ $ \bigwedge \limits_{b=1}^{n}  ( \neg X_{a,b,c} \vee P_{a,c} ) \wedge
 (\neg P_{a,c} \vee  (\bigvee \limits_{b=1}^{n}  X_{a,b,c}) )$ ] 


\subsubsection{Contrainte : Au plus un même musicien dans un même groupe}


\underline{Pour tout} groupe $c \in G$, \underline{pour tout} musicien $a \in M $, \underline{il n'existe pas} deux instruments différents b,b' $\in I$  tel que $b < b'$ et $X_{a,b,c}$
\underline{et} $X_{a,b',c}$.


$\bigwedge \limits_{c=1}^{k}$ $\bigwedge \limits_{a=1}^{m}$ $\neg$ [ $\bigvee \limits_{b,b' \in (1..n); b < b'}$ ( $X_{a,b,c}$ $\wedge$ $X_{a,b',c}$ )]


$\equiv$ $\bigwedge \limits_{c=1}^{k}$ $\bigwedge \limits_{a=1}^{m}$  $\bigwedge \limits_{b,b' \in (1..n); b < b'}$ ( $\neg X_{a,b,c}$ $\vee $ $ \neg X_{a,b',c}$ )


\subsubsection{Contrainte : Au plus un musicien pour le même instrument au sein d'un même groupe}
 
\underline{Pour tout} groupe $c \in G$, \underline{pour tout} instrument $b \in I$ \underline{il n'existe pas} a,a' $\in M$  tel que $a < a'$ et $X_{a,b,c}$
\underline{et} $X_{a',b,c}$ . 
 
$\bigwedge \limits_{c=1}^{k}$ $\bigwedge \limits_{b=1}^{n}$ $\neg$ [ $\bigvee \limits_{a,a' \in (1..m); a < a'}$ ( $X_{a,b,c}$ $\wedge$ $X_{a',b,c}$ )]


$\equiv$ $\bigwedge \limits_{c=1}^{k}$ $\bigwedge \limits_{b=1}^{n}$  $\bigwedge \limits_{a,a' \in (1..m); a < a'}$ ( $\neg X_{a,b,c}$ $\vee $ $ \neg X_{a',b,c}$ ) 
 
 
\subsubsection{Contrainte : Un groupe doit être soit complet soit vide} 
 
\begin{enumerate}

 \item Si un musicien joue d'un instrument dans un groupe, alors ce groupe est actif.
 
    \underline{Pour tout} groupe $c \in G$, \underline{pour tout} musicien $a \in M$ \underline{pour tout} instrument $b \in I$  \underline{si} $X_{a,b,c}$ \underline{alors} $A_c$.
    
    $ \bigwedge \limits_{c=1}^{k} \bigwedge \limits_{a=1}^{m} \bigwedge \limits_{b=1}^{n}$ ( $ X_{a,b,c} \rightarrow A_c$ )
    
    $\equiv \bigwedge \limits_{c=1}^{k} \bigwedge \limits_{a=1}^{m} \bigwedge \limits_{b=1}^{n}$ ( $ \neg  X_{a,b,c} \vee A_c$ )

 
 
 \item \underline{Si} un groupe est actif, \underline {alors} ce groupe est complet.
 
    \underline{Pour tout} groupe $c \in G$, \underline{si} $A_c$ \underline{alors} \underline{pour tout} instrument $b \in I$ \underline{il existe} un musicien $a \in M$
    tel que $X_{a,b,c}$.
    
    
    $ \bigwedge \limits_{c=1}^{k} [ A_c \rightarrow (  \bigwedge \limits_{b=1}^{n} \bigvee \limits_{a=1}^{m} X_{a,b,c} ) ]$
    
    $\equiv \bigwedge \limits_{c=1}^{k} [ \neg A_c \vee (  \bigwedge \limits_{b=1}^{n} \bigvee \limits_{a=1}^{m} X_{a,b,c} ) ]$
    
    $\equiv \bigwedge \limits_{c=1}^{k} \bigwedge \limits_{b=1}^{n}  [ \neg A_c \vee ( \bigvee \limits_{a=1}^{m} X_{a,b,c} ) ]$
 
\end{enumerate}

\subsubsection{Contrainte : Un musicien est dans au plus n groupes ($Max_a$)}

\underline{Pour tout} musicien $a \in M$, \underline{il n'existe pas} des groupes $(c^1,c^2,..c^{n+1}) \in G$ tel que $c^1 < c^2 <...<c^{n+1}$ tel que $P_{a,c^1}$ \underline{et} $P_{a,c^2}$
\underline{et} ... \underline{et} $P_{a,c^{n+1}}$
 
$ \bigwedge \limits_{a=1}^{m} \neg ( \bigvee \limits_{c^1,c^2,..c^{n+1} \in G; c^1 < c^2 <...<c^{n+1}} ( P_{a,c^1} \wedge P_{a,c^2} \wedge ... \wedge P_{a,c^{n+1}} ))$

\subsubsection{Contrainte : Un musicien ne peut pas apparaître dans un groupe pour un instrument qu'il ne sait pas jouer}


\underline{Pour tout} musicien $a \in M $, \underline{pour tout} groupe $c \in G$ , \underline{pour tout} instrument $b \notin $ instumentsPlayed(a)  tel que $\neg X_{a,b,c}$


$\bigwedge \limits_{a=1}^{m}$  $\bigwedge \limits_{c=1}^{k}$ $\bigwedge \limits_{b \notin instumentsPlayed(a) }$  $\neg X_{a,b,c}$


\subsection{Nouvelles contraintes : }

\subsubsection{Contrainte : Un musicien est dans au moins un groupe (pour un instrument qu'il sait jouer) ou pour chanter}


\underline{Pour tout} musicien $a \in M $, \underline{il existe} un groupe $c \in G$ , (\underline{il existe} un instrument $b \in$ instumentsPlayed(a)  tel que $X_{a,b,c}$) \underline{ou} $C_{a,c} $


$\bigwedge \limits_{a=1}^{m}$ [ $\bigvee \limits_{c=1}^{k}$ $ ( ( \bigvee \limits_{b \in instumentsPlayed(a) }$  $X_{a,b,c}) \vee C_{a,c} ) $ ]


\subsubsection{Contrainte : Au moins un chanteur par groupe}

\underline{Pour tout} groupe $c \in G$, \underline{si} $A_c$ \underline{alors} \underline{il existe} un musicien $a \in M$ tel que $C_{a,c}$ .

$ \bigwedge \limits_{c=1}^{k} ( A_c \rightarrow \bigvee \limits_{a=1}^{m} C_{a,c} ) $ 
$ \equiv \bigwedge \limits_{c=1}^{k} ( \neg A_c \vee \bigvee \limits_{a=1}^{m} C_{a,c} )  $

\subsubsection{Contrainte : Seuls les musiciens qui maitrisent le chant peuvent chanter au sein d'un groupe }

\underline{Pour tout} groupe $c \in G$, \underline{pour tout} musicien $a \notin M'$ tel que $\neg C_{a,c}$ .

$ \bigwedge \limits_{c=1}^{k}  \bigwedge \limits_{a \notin M'} \neg C_{a,c}$


\subsection{Impact de la modification sur la présentation du problème}

Le fait qu'un musicien peut maintenant chanter (le chant est considéré comme un 'instrument' spécial) entraine que pour qu'un musicien soit au moins une fois dans un groupe il doit soit jouer un instrument soit chanter.
Nous ne considérons pas le chant comme un instrument. Nous avons donc introduit une nouvelle variable : $C_{a,c} $ qui est vraie uniquement si le musicien a chante dans le groupe c.

Cela amene à la modification de la contrainte 'Un musicien est dans au moins un groupe (pour un instrument qu'il sait jouer)' ( nous rajoutons 'ou pour chanter' )  

$\bigwedge \limits_{a=1}^{m}$ [ $\bigvee \limits_{c=1}^{k}$ $ (  \bigvee \limits_{b \in instumentsPlayed(a) }$  $X_{a,b,c}) $ ]

devient alors

$\bigwedge \limits_{a=1}^{m}$ [ $\bigvee \limits_{c=1}^{k}$ $ ( ( \bigvee \limits_{b \in instumentsPlayed(a) }$  $X_{a,b,c}) \vee C_{a,c} ) $ ]

Il faut également introduire les contraintes qu'il y a au moins un chanteur par groupe et que seuls les musiciens qui maitrisent le chant peuvent chanter dans un groupe.

\end{document}
